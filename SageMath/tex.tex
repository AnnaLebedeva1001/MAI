\documentclass[14pt]{extreport}
\usepackage[left=2.5cm, right=1.5cm, top=2.5cm, bottom=2.5cm]{geometry}
\usepackage[utf8]{inputenc}
\usepackage{indentfirst}
\usepackage[T2A]{fontenc}
\usepackage[english,russian]{babel}
\usepackage{amsmath}
\usepackage{amssymb}
\usepackage[usenames]{color}
%\usepackage{hyperref}
\usepackage{sagetex}
\setlength{\sagetexindent}{10ex}
\linespread{1.3}
%\renewcommand{\baselinestretch}{1.5}
\renewcommand{\thesection}{\number\numexpr\value{section}-1\relax}
\renewcommand{\thesubsection}{\thesection.\number\numexpr\value{subsection}-1\relax}
\renewcommand{\thesubsubsection}{\thesubsection.\number\numexpr\value{subsubsection}-1\relax}
\setcounter{secnumdepth}{1}
\setcounter{chapter}{1}
\setcounter{section}{1}
\setcounter{page}{2}

\begin{document}
\tableofcontents

\newpage
\section{Исследование графиков функций}

\begin{sagesilent}
    f(x) = x - ln(x) - 2
\end{sagesilent}

Для функции $f(x) = \sage{f(x)}$ найти:

\subsubsection{Область определения функции}

Область определения функции - это множество точек, на каждой из которых функция имеет значение. Рассмотрим
функцию $f(x)$: функция $y = \log{(x)}$ не имеет значений при $x \in (-\infty;0]$. Поэтому
область определениия функции $f(x)$: $D(f(x)) = (-\infty;\infty)\setminus(-\infty;0]$ или $(0;\infty)$.

\subsubsection{Является ли функция четной или нечетной, является ли периодической}

Исследуем функцию $f(x)$ на четность: $f(x) = x - \log{(x)} - 2, ~D(f(x)) = (0;\infty);~
f(-x) = -x - \log{(-x)} - 2,~D(f(-x)) = (-\infty;0)$. Так как области определения функций $f(x)$ и $f(-x)$
несовпадают, функция $f(x)$ не является ни четной, ни нечетной.

Исследуем функцию $f(x)$ на периодичность: известно, что $y = \log{(x)}, y = x$ -- непериодические функции,
поэтому, основываясь на свойстве суммы периодических функций, $f(x) = x - \log{(x)} - 2$ -- непериодическая
функция.

\subsubsection{Точки пересечения графика с осями координат}

Найдем точки пересечения графика функции $f(x)$ с осью ординат: $f(0) = 0 - \log{(0)} - 2$, но $\log{(0)} =
\infty$, следовательно, $f(0)$ -- асимптота и функция $f(x)$ не имеет пересечений с осью ординат.

Найдем точки пересечения графика функции $f(x)$ с осью абсцисс. Для этого решим уравнение: $x - \log{(x)} -
2 = 0$, откуда $x_1 \approx 0.15859433956303934, x_2 \approx 3.146193220$.

\subsubsection{Промежутки знакопостоянства}

Используя ранее найденные область определения функции $D(f(x)) = (0;\infty)$ и точки пересечения с осью
абсцисс $x_1 \approx 0.1585, x_2 \approx 3.1461$, определим промежутки знакопостоянства функции $f(x)$. 
При $x \in (0; 0.1585)\cup (3.1461; \infty)$ $f(x) > 0$, а при $x \in (0.1585; 3.1461)$ $f(x) < 0$, так как
проходя через точки $x_1, x_2$ функция меняет знак, а при $f(t)$, где $t \in (0.1585; 3.1461),$ например,
при $t = 1, f(x) = -1 < 0$.

\subsubsection{Промежутки возрастания и убывания}

Производная функции $f(x) = x - \log{(x)} - 2$: $f'(x) = -\frac{1}{x} + 1$. Решим систему неравенств:
$\begin{cases}x > 0\\f'(x) = -\frac{1}{x} + 1 > 0\end{cases}$, получим $x > 1$ ---
промежуток возрастания функции. Теперь найдем решение системы: $\begin{cases}x > 0\\f'(x) =
-\frac{1}{x} + 1 < 0\end{cases}$, получим $x \in (0; 1)$ --- промежуток убывания функции.

\subsubsection{Точки экстремума и значения в этих точках}

Найдем критические точки, принадлежащие области определения $f(x)$, для этого решим систему:
$\begin{cases}x > 0\\f'(x) = -\frac{1}{x} + 1 = 0\end{cases}$ и получим $x = 1$ --- точка
минимума $f(x)$, поскольку, проходя через нее, функция меняет свой характер с убывания на возрастание.
В точке минимума $x = 1$ функция принимает значение $f(1) = 1 - \log{(1)} - 2 = -1$.

\subsubsection{Исследовать поведение функции в окрестности «особых» точек и при больших по модулю $x$.}

Имеем одну особую точку $x = 0$. В ее правой окрестности функция $f(x) = x - \log{(x)} - 2$ убывает, а в левой окрестности --- функция неопределена. При больших по модулю положительных $x: f(x)\to+\infty$, при
отрицательных $x$ --- функция не существует.

\subsubsection{Непрерывность. Наличие точек разрыва и их классификация}

Функция $f(x) = x - \log{(x)} - 2$ непрерывна на всей области определения $D(f(x)) = (0;\infty)$.

\subsubsection{Асимптоты}

\begin{enumerate}
\item Вертикальные: функция $f(x)$ имеет разрыв в точке $x = 0$, следовательно, $x = 0$ - вертикальная
асимптота.

\item Наклонные: приведем функцию $f(x)$ к виду $f(x) = k*x + b,\\
k = \lim_{x\to\infty}\frac{f(x)}{x} = \lim_{x\to\infty}(1 - \frac{2}{x} - \frac{\log{(x)}}{x}) = 1, b =
lim_{x\to\infty}[f(x) - kx] = x - x - \log{(x)} - 2 = -\infty$, следовательно, у функции $f(x)$ нет
наклонных асимптот.

\item Горизонтальная: так как коэффициент $b = -\infty$, функция $f(x)$ не имеет горизонтальных асимптот.
\end{enumerate}

\subsubsection{Построить график функции, асимптоты}

График функции $f(x)$ на промежутке $[-10;10]$ на рисунке 1.

\begin{sagesilent}
x_coords = [0*t for t in srange(-10, 10, 0.02)]
y_coords = [t for t in srange(-10, 10, 0.02)]
plot(f(x), -10, 10) + list_plot(list(zip(x_coords, y_coords)))
\end{sagesilent}
\sageplot{plot(f(x), -10, 10) + list_plot(list(zip(x_coords, y_coords)))}\eqn(1)

\section{Решение системы линейных алгебраических\\уравнений}

\subsubsection{Метод Крамера}

Найдем решение системы линейных алгебраических уравнений (СЛАУ)\\$\begin{cases}x_1 + 2x_2 + 4x_3 = 31\\
5x_1 + x_2 + 2x_3 = 29\\ 3x_1 - x_2 + x_3 = 10\end{cases}$ методом Крамера.
Имеем матрицу $A = \begin{pmatrix}1&2&4\\5&1&2\\3&-1&1\end{pmatrix}$ и столбец $b = \begin{pmatrix}
31\\29\\10\end{pmatrix}$. $D = \det{(A)} = \begin{vmatrix}
1&2&4\\5&1&2\\3&-1&1\end{vmatrix} = -27$, --- матрица $A$ невырожденная, следоваельно, существует
единственное решение. Найдем его, для этого составим матрицы 
$A_1 = \begin{pmatrix}31&2&4\\29&1&2\\10&-1&1\end{pmatrix},
A_2 = \begin{pmatrix}1&31&4\\5&29&2\\3&10&1\end{pmatrix},
A_3 = \begin{pmatrix}1&2&31\\5&1&29 \\3&-1&10\end{pmatrix}$ и найдем их определители
$d_1 = \det{(A_1)} = \begin{vmatrix}31&2&4\\29&1&2\\10&-1&1\end{vmatrix} =
-81, d_2 = \det{(A_2)} = \begin{vmatrix}1&31&4\\5&29&2\\3&10&1\end{vmatrix} = -108, d_3 = \det{(A_3)} =
\begin{vmatrix}1&2&31\\5&1&29 \\3&-1&10\end{vmatrix} = -135$. Тогда $\begin{cases}x_1 = \frac{d_1}
{D} = 3\\ x_2 = \frac{d_2}{D} = 4 \\ x_3 = \frac{d_3}{D} =5\end{cases}$.


\subsubsection{Метод Гаусса}

Решим СЛАУ $\begin{cases}x_1 + 2x_2 + 4x_3 = 31\\ 5x_1 + x_2 + 2x_3 = 29\\ 3x_1 - x_2 + x_3 = 10\end{cases}$
методом Гаусса. Составим матрицу $A = \begin{pmatrix}1&2&4\\5&1&2\\3&-1&1\end{pmatrix}$ и
расширенную матрицу $B = \begin{pmatrix}1&2&4&31\\5&1&2&29\\3&-1&1&10\end{pmatrix}$.
Проверим совместность по теореме Кронеккера-Капелли: ранг матрицы коэффициентов $rg{(A)} =
3$, ранг расширенной матрицы $rg{(B)} = 3$. Ранги совпали, следовательно, СЛАУ --- совместна, а значит
имеет единственное решение. Найдем его: диаганализируем расширенную матрицу $B$ и получим
$\begin{pmatrix}1&2&4&31\\5&1&2&29\\3&-1&1&10\end{pmatrix}$ \Rightarrow\\
$\begin{pmatrix}1&0&0&3\\0&1&0&4\\0&0&1&5\end{pmatrix}$ \Rightarrow
$\begin{cases}x_1 = 3\\x_2 = 4\\x_3 = 5\end{cases}$.


\section{Решение матричных уравнений}

$\frac{1}{2}\begin{pmatrix}-2&4&-6\\2&0&-8\\4&4&2\end{pmatrix}^2 = \begin{pmatrix}2&1&1\\-1&-1&4\\1&0&3
\end{pmatrix} + 3X\begin{pmatrix}3&8&-9\\-1&1&4\\1&0&2\end{pmatrix}$;

${X}\begin{pmatrix}3&8&-9\\-1&1&4\\1&0&2 \end{pmatrix} = \begin{pmatrix}-\frac{8}{3}&-\frac{17}{3}&-\frac{17}{3}\\-\frac{17}{3}&-\frac{11}{3}&-6\\1&4&-\frac{29}{3}\end{pmatrix}$;

$\begin{pmatrix}-\frac{8}{3}&-\frac{17}{3}&-\frac{17}{3}\\-\frac{17}{3}&-\frac{11}{3}&-6\\1&4&-\frac{29}
{3}\end{pmatrix}^{-1}\cdot{X} = \begin{pmatrix}3&8&-9\\-1&1&4\\1&0&2\end{pmatrix}^{-1}$;

${X} = \begin{pmatrix}-\frac{8}{3}&-\frac{17}{3}&-\frac{17}{3}\\ -\frac{17}{3}&-\frac{11}{3}&-6\\1&4&-\frac{29}{3}\end{pmatrix}\cdot\begin{pmatrix}3&8&-9\\-1&1&4\\1&0&2 \end{pmatrix}^{-1} = \begin{pmatrix}\frac{2}{3}&1&-\frac{11}{3}\\\frac{203}{93}&\frac{481}{93}&-\frac{655}{93}\\\frac{37}{31}&\frac{68}{93}&-\frac{172}{93}\end{pmatrix}$.


\section{Решение алгебраических уравнений третей степени}

Задано уравнение третьей степени: $9x^3 + 7x^2 - 7x - 2 = 0$.

\subsubsection{Решение методом Кардано}

Имеем $\begin{cases}a = \frac{7}{9}\\ b = -\frac{7}{9}\\c = -\frac{2}{9}\end{cases}$, тогда 
$\begin{cases}p = -{{a^2}\over{3}} + b =
-{{7^2}\over{3}}+(-7) =-\frac{238}{243}\\q = 2\cdot({{a}\over{3}})^3 - {{ab}\over{3}}+ c =
2\cdot({{7}\over{3}})^3 - {{7\cdot(-7)}\over{3}}-2 =\frac{281}{19683}\end{cases}$

$Q = (\frac{p}{3})^3 +(\frac{q}{2})^2 = (\frac{-70}{9})^3 + (\frac{1073}{54})^2 = -\frac{8207}{236196}$;

$A = \sqrt[3]{-{{q}\over{2}}+\sqrt{Q}} = \sqrt[3]{-{{1073}\over{54}}+\sqrt{-\frac{8173}{108}}} =
\sqrt[3]{-{{1073}\over{54}}+\sqrt{-\frac{\sqrt{24519}\cdot i}{18}}}$;

$B = \sqrt[3]{-{{q}\over{2}}-\sqrt{Q}} = \sqrt[3]{-{{1073}\over{54}}-\sqrt{-\frac{8173}{108}}} =
\sqrt[3]{-{{1073}\over{54}}-\sqrt{-\frac{\sqrt{24519}\cdot i}{18}}}$;

$y_1 = A + B = \sqrt[3]{-{{1073}\over{54}}-\sqrt{-\frac{\sqrt{24519}\cdot i}{18}}} +
\sqrt[3]{-{{1073}\over{54}}+\sqrt{-\frac{\sqrt{24519}\cdot i}{18}}}$;

$y_2 = -\frac{A + B}{2} + i\frac{A - B}{2}\sqrt{3} =
-\frac{\sqrt[3]{-{{1073}\over{54}}-\sqrt{-\frac{\sqrt{24519}\cdot i}{18}}} +
\sqrt[3]{-{{1073}\over{54}}+\sqrt{-\frac{\sqrt{24519}\cdot i}{18}}}}{2} +\\
i\frac{\sqrt[3]{-{{1073}\over{54}}-\sqrt{-\frac{\sqrt{24519}\cdot i}{18}}} -
\sqrt[3]{-{{1073}\over{54}}+\sqrt{-\frac{\sqrt{24519}\cdot i}{18}}}}{2}\sqrt{3}$;

$y_3 = -\frac{A + B}{2} - i\frac{A - B}{2}\sqrt{3} =
-\frac{\sqrt[3]{-{{1073}\over{54}}-\sqrt{-\frac{\sqrt{24519}\cdot i}{18}}} +
\sqrt[3]{-{{1073}\over{54}}+\sqrt{-\frac{\sqrt{24519}\cdot i}{18}}}}{2} -\\
i\frac{\sqrt[3]{-{{1073}\over{54}}-\sqrt{-\frac{\sqrt{24519}\cdot i}{18}}} -
\sqrt[3]{-{{1073}\over{54}}+\sqrt{-\frac{\sqrt{24519}\cdot i}{18}}}}{2}\sqrt{3}$;

$x_1 = y_1 - \frac{a}{3} = - \frac{7}{27} + \left(- \frac{\sqrt[3]{- \frac{281}{39366} + \sqrt{8207}
\frac{i}{486}}}{2} \left(1 + \sqrt{3} i\right) - \frac{119 \left(1 + \sqrt{3} \left(- i\right)\right)}{729
\sqrt[3]{- \frac{281}{39366} + \sqrt{8207} \frac{i}{486}}}\right)$;

$x_2 = y_2 - \frac{a}{3} = x = - \frac{7}{27} + \left(- \frac{119 \left(1 + \sqrt{3} i\right)}{729
\sqrt[3]{- \frac{281}{39366} + \sqrt{8207} \frac{i}{486}}} + - \frac{\sqrt[3]{- \frac{281}{39366} +
\sqrt{8207} \frac{i}{486}}}{2} \left(1 + \sqrt{3} \left(- i\right)\right)\right)$;

$x_3 = y_3 - \frac{a}{3} = x = - \frac{7}{27} + \left(\frac{238}{729 \sqrt[3]{- \frac{281}{39366} +
\sqrt{8207} \frac{i}{486}}} + \sqrt[3]{- \frac{281}{39366} + \sqrt{8207} \frac{i}{486}}\right)$.


\subsubsection{График функции}

\begin{sagesilent}
xx1 = find_root(9*x^3 + 7*x^2 - 7*x - 2 == 0, -2, -1)
xx2 = find_root(9*x^3 + 7*x^2 - 7*x - 2 == 0, -1, 0)
xx3 = find_root(9*x^3 + 7*x^2 - 7*x - 2 == 0, 0, 2)

g = 9*x^3 + 7*x^2 - 7*x - 2
plot(g, -2, 2) + list_plot([(xx1, 0), (xx2, 0), (xx3, 0)], size=65, rgbcolor="red")
\end{sagesilent}
\sageplot{plot(g, -2, 2) + list_plot([(xx1, 0), (xx2, 0), (xx3, 0)], size=65, rgbcolor="red")}\eqn(2)

\subsubsection{Представление комплексных корней}

\begin{sagesilent}
# 4.2 График функции
a = 7/9
b = -7/9
c = -2/9
p = -(a^2)/3 + b
q = 2 * (a/3)^3 - (a*b/3) + c
Q = (p/3)^3 + (q/2)^2
A = (-(q/2)+sqrt(Q))^(1/3)
B = (-(q/2)-sqrt(Q))^(1/3)
y1 = A + B
y2 = -(A + B)/2 + i*(A - B)/2*sqrt(3)
y3 = -(A + B)/2 - i*(A - B)/2*sqrt(3)
x1 = y1 - a/3
x2 = y2 - a/3
x3 = y3 - a/3

xx1 = find_root(9*x^3 + 7*x^2 - 7*x - 2 == 0, -2, -1)
xx2 = find_root(9*x^3 + 7*x^2 - 7*x - 2 == 0, -1, 0)
xx3 = find_root(9*x^3 + 7*x^2 - 7*x - 2 == 0, 0, 2)
# Алгебраическая форма комплексых корней
x1
x2
x3
# Тригонометрическая форма комплексых корней
x1.real().simplify() + x1.imag().simplify()
x2.real().simplify() + x2.imag().simplify()
x3.real().simplify() + x3.imag().simplify()
# Экспоненциальная форма комплексых корней
exp_polar(x1)
exp_polar(x2)
exp_polar(x3)

\end{sagesilent}

Алгебраическая форма:\\
$x_1 = \sage{x1}$\\
$x_2 = \sage{x2}$\\
$x_3 = \sage{x3}$\\\\

Тригонометрическая форма:\\
$x_1 = \sage{x1.real().simplify() + x1.imag().simplify()}$\\
$x_2 = \sage{x2.real().simplify() + x2.imag().simplify()}$\\
$x_3 = \sage{x3.real().simplify() + x3.imag().simplify()}$\\

Экспоненциальная форма:\\
$x_1 = \sage{exp_polar(x1)}$\\
$x_2 = \sage{exp_polar(x2)}$\\
$x_3 = \sage{exp_polar(x3)}$

\section{Решение алгебраических уравнений четвертой степени}

\subsubsection{Решение методом Феррари}

Имеем уравнение 4-й вида:
$Ax^4+Bx^3+Cx^2+Dx+E = 0$, в котором $ \begin{cases} A = -3\\B = 7\\C = -2\\D = 15\\E = -5 \end{cases} $

\begin{sagesilent}
A = -3
B = 7
C = -2
D = 15
E = -5
a = -3* B^2/(8*A^2)+ C/A
b = B^3/(8*A^3) - B*C/(2*A^2) + D/A
y = -3*B^4/(256*A^4) - B^2*C/(16*A^3) - B*D/(4*A^2) + E/A
P = -A^2/12 - y
Q = -a^3/108 + a/3 - b^2/8
R1 = -Q/2 + sqrt(Q^2/4 + P^3/27)
R2 = -Q/2 - sqrt(Q^2/4 + P^3/27)
U = 1/144*(2*sqrt(24345158095013/2) + 6948451)^(1/3)
y = -(5/6)*a + U + (-P/3*U)
W = sqrt(a + 2 * y)
x1 = -1/6*(1/2)^(1/3)*(3*sqrt(835)*sqrt(3) + 151)^(1/3)*(I*sqrt(3) + 1) - 4/3*(1/2)^(2/3)*(-I*sqrt(3) + 1)/(3*sqrt(835)*sqrt(3) + 151)^(1/3) + 2/3
x2 = -1/6*(1/2)^(1/3)*(3*sqrt(835)*sqrt(3) + 151)^(1/3)*(-I*sqrt(3) + 1) - 4/3*(1/2)^(2/3)*(I*sqrt(3) + 1)/(3*sqrt(835)*sqrt(3) + 151)^(1/3) + 2/3
x3 = 1/3*(1/2)^(1/3)*(3*sqrt(835)*sqrt(3) + 151)^(1/3) + 8/3*(1/2)^(2/3)/(3*sqrt(835)*sqrt(3) + 151)^(1/3) + 2/3
x4 = 1/3
\end{sagesilent}

Его решение может быть найдено из следующих выражений:
$ \alpha = - {3 B^2 \over 8 A^2} + {C \over A} = \sage{a}$,
$\beta = {B^3 \over 8 A^3} - {B C \over 2 A^2} + {D \over A} = \sage{b}$,\\
$\gamma = -{3 B^4 \over 256 A^4} + {B^2 C \over 16 A^3} - {B D \over 4 A^2} + {E \over A} = \sage{y}$
$P = - {\alpha^2 \over 12} - \gamma = \sage{P}$,
$Q = - {\alpha^3 \over 108} + {\alpha \gamma \over 3} - {\beta^2 \over 8} = \sage{Q}$,
$R = -{Q\over 2} \pm \sqrt{{Q^{2}\over 4}+{P^{3}\over 27}} = \sage{R1}$,
(любой знак квадратного корня подойдёт)
$U = \sqrt[3]{R} = \frac{1}{144}(2\sqrt{\frac{24345158095013}{2}} + 6948451)^\frac{1}{3}$,
(три комплексных корня, один из которых подойдёт)
$y = - {5 \over 6} \alpha +U + \begin{cases}U=0 &\to -\sqrt[3]{Q}\\U\ne 0 &\to {-P\over 3U}\end{cases}
= \sage{y}$,
$W=\sqrt{ \alpha + 2 y} = \sage{W}$,
$x = - {B \over 4 A} + { \pm_s  W \pm_t \sqrt{-\left(3\alpha + 2 y \pm_s {2\beta\over W} \right) }\over 2}$.
Два $±s$ — один и тот же знак при нахождении конкретного x, при этом $±t$ будет другим или тем же.
Все корни x можно найти при всех четырёх комбинациях знаков ±$s$ и ±$t$: «$+$,$+$»; «$+$,$−$»; «$−$,$+$» и
«$-$,$-$».
Таким образом, получаем 4 корня:\\
$\begin{cases}x_1 = \sage{x1}\\x_2 = \sage{x2}\\x_3 = \sage{x3}\\x_4 = \sage{x4}\end{cases}$

\subsubsection{График функции}
\begin{sagesilent}
g = -3*x^4 + 7*x^3 - 2*x^2 + 15*x - 5
xxx1 = find_root(g == 0, 0, 1)
xxx2 = find_root(g == 0, 2, 3)
plot(g, -2, 3) + list_plot([(xxx1, 0), (xxx2, 0)], size=65, rgbcolor="red")
\end{sagesilent}
\sageplot{plot(g, -2, 3) + list_plot([(xxx1, 0), (xxx2, 0)], size=65, rgbcolor="red")}\eqn(3)

\subsubsection{Представление комплексных корней}
%\begin{sagesilent}
%\end{sagesilent}

Алгебраическая форма:\\
$x_1 = \sage{x1}$\\
$x_2 = \sage{x2}$\\

Тригонометрическая форма:\\
$x_1 = \sage{exp_polar(x1).real().simplify() + exp_polar(x1).imag().simplify()}$\\
$x_2 = \sage{exp_polar(x2).real().simplify() + exp_polar(x2).imag().simplify()}$\\

Экспоненциальная форма:\\
$x_1 = \sage{exp_polar(x1)}$\\
$x_2 = \sage{exp_polar(x2)}$\\

\section{Нахождение наибольшего общего делителя двух полиномов}
Даны полиномы: $f(x) = x^4 + 2x^3 - x^2 - 6x - 4$, $g(x) = 2x^3 - 2x - 1$. Применим к ним алгоритм Евклида
для нахождения их наименьшего общего делителя (НОД), для этого разделим первый на второй и возьмем остаток:
$t_1(x) = \frac{f(x)}{g(x)} = -\frac{7}{2}x - 3$, $t_2(x) = \frac{g(x)}{t_1(x)} = -\frac{187}{343}$,
таким образом, НОД полиномов равен 1.

\section{Линейные преобразования и характеристическое уравнение матрицы}

Дано преобразование и базис: $\begin{cases}e_1` = e_1 + 3e_2\\e_2` = 2e_1 + 3e_2\\e_3` = 2e_2 - 3e_3\end{cases}$,
$A = \begin{vmatrix}4&-1&-2\\2&1&-2\\1&-1&1\end{vmatrix}$. Для того чтобы перевести преобразование в иной базис
составим матрицу перехода $S = \begin{pmatrix}2&4&1\\2&1&0\\1&0&1\end{pmatrix}$. Матрица $A$ в новом базисе:
$A` = S^{-1}AS = \frac{1}{7}\begin{pmatrix}13&24&0\\2&15&0\\-6&-3&2\end{pmatrix}$, таким образом мы составили
новый базис: \begin{cases}e_1`=\frac{13}{7}e_1 + \frac{24}{7}\\e_2` = \frac{2}{7}e_1 + \frac{15}{7}e_2\\
e_3` = -\frac{6}{7}e_1 - \frac{3}{7}e_2 + \frac{2}{7}e_3\end{cases}.

Теперь найдем характеристические полиномы для $A$ и $A`$: $\det{(A - \lambda\cdot E)} = \lambda^3 - 6\lambda^2 +
11\lambda - 6; \det{(A` - \lambda\cdot E)} = \lambda^3 - 6\lambda^2 + 11\lambda - 6$ --- они совпали, следовательно,
найденный базис эквивалентен исходному. Теперь найдем корни характеристического уравнения:
$\lambda^3 - 6\lambda^2 + 11\lambda - 6 = 0$, откуда получим: $\begin{cases}\lambda_1 = 1\\
\lambda_2 = 3\\\lambda_3 = 3\end{cases}$ --- все они собственные числа для матриц $A$ и $A`$.

\section{Упрощение уравнений фигур второго порядка на плоскости}

Имеем уравнение: $-9x^2 + 7y^2 - 3z^2 + 8yz - 4x + 9y - 10 = 0$. Составим по нему матрицу $A = \begin{pmatrix}
-9&0&0\\0&7&4\\0&4&-3\end{pmatrix}$ и вектор $a = \begin{pmatrix}-4&9&0\end{pmatrix}^T$. Вычислим ортогональные варианты:
$\tau_1 = a_{11} + a_{22} + a_{33} = -5$, $\tau_2 = \begin{vmatrix}a_{11}&a_{12}\\a_{21}&a_{22}\end{vmatrix}
+ \begin{vmatrix}a_{11}&a_{13}\\a_{31}&a_{33}\end{vmatrix} + \begin{vmatrix}a_{22}&a_{23}\\a_{32}&a_{33}
\end{vmatrix} = -73$; $\det{(A)} = 333$, $\begin{vmatrix}-9&0&0&-2\\0&7&4&4.5\\
0&4&-3&0\\-2&4.5&0&-10\end{vmatrix} = -\frac{16211}{4}$. По найденным характеристикам можно определить двуполосный
гиперболоид.

Приведем уравнение к каноническому виду. Получим:
$\frac{X^2}{\frac{16211}{11988}} + \frac{Y^2}{\frac{16211}{1332(\sqrt{41} - 21}} - \frac{Z^2}{\frac{16211}{1332(\sqrt{41} + 21}} = -1$.

\begin{sagesilent}
u = integral((1 + ln(x - 1))/(x - 1), x, exp(1) + 1, exp(2) + 1)
q = (1 + ln(x - 1))/(x - 1)
plot(q, 0, 15)
\end{sagesilent}

\section{Численные методы: интегралы}

\sageplot{plot(q, 0, 15)}

Вычислим интеграл: $\sage{integral((1 + ln(x - 1))/(x - 1), x, exp(1) + 1, exp(2) + 1)}$.

\subsubsection{Методы трапеции и прямоугольников вычисления площади}
TrapMet

step: 1; curr res: 0.034219; n = 100; a = 8.389056; b = 3.718282

step: 2; curr res: 0.068148; n = 100; a = 8.389056; b = 3.718282

step: 3; curr res: 0.101792; n = 100; a = 8.389056; b = 3.718282

step: 4; curr res: 0.135155; n = 100; a = 8.389056; b = 3.718282

step: 5; curr res: 0.168243; n = 100; a = 8.389056; b = 3.718282

step: 6; curr res: 0.201060; n = 100; a = 8.389056; b = 3.718282

step: 7; curr res: 0.233610; n = 100; a = 8.389056; b = 3.718282

step: 8; curr res: 0.265898; n = 100; a = 8.389056; b = 3.718282

step: 9; curr res: 0.297928; n = 100; a = 8.389056; b = 3.718282

step: 10; curr res: 0.329705; n = 100; a = 8.389056; b = 3.718282

step: 11; curr res: 0.361234; n = 100; a = 8.389056; b = 3.718282

step: 12; curr res: 0.392517; n = 100; a = 8.389056; b = 3.718282

step: 13; curr res: 0.423559; n = 100; a = 8.389056; b = 3.718282

step: 14; curr res: 0.454364; n = 100; a = 8.389056; b = 3.718282

step: 15; curr res: 0.484935; n = 100; a = 8.389056; b = 3.718282

step: 16; curr res: 0.515278; n = 100; a = 8.389056; b = 3.718282

step: 17; curr res: 0.545394; n = 100; a = 8.389056; b = 3.718282

step: 18; curr res: 0.575289; n = 100; a = 8.389056; b = 3.718282

step: 19; curr res: 0.604964; n = 100; a = 8.389056; b = 3.718282

step: 20; curr res: 0.634425; n = 100; a = 8.389056; b = 3.718282

Метод трапеций вернул результат: 2.50001794439142 > настоящего результата

RectMet

step: 1; curr res: 0.034219; n = 100; a = 8.389056; b = 3.718282

step: 2; curr res: 0.068147; n = 100; a = 8.389056; b = 3.718282

step: 3; curr res: 0.101790; n = 100; a = 8.389056; b = 3.718282

step: 4; curr res: 0.135153; n = 100; a = 8.389056; b = 3.718282

step: 5; curr res: 0.168240; n = 100; a = 8.389056; b = 3.718282

step: 6; curr res: 0.201056; n = 100; a = 8.389056; b = 3.718282

step: 7; curr res: 0.233605; n = 100; a = 8.389056; b = 3.718282

step: 8; curr res: 0.265893; n = 100; a = 8.389056; b = 3.718282

step: 9; curr res: 0.297923; n = 100; a = 8.389056; b = 3.718282

step: 10; curr res: 0.329700; n = 100; a = 8.389056; b = 3.718282

step: 11; curr res: 0.361227; n = 100; a = 8.389056; b = 3.718282

step: 12; curr res: 0.392510; n = 100; a = 8.389056; b = 3.718282

step: 13; curr res: 0.423551; n = 100; a = 8.389056; b = 3.718282

step: 14; curr res: 0.454356; n = 100; a = 8.389056; b = 3.718282

step: 15; curr res: 0.484927; n = 100; a = 8.389056; b = 3.718282

step: 16; curr res: 0.515269; n = 100; a = 8.389056; b = 3.718282

step: 17; curr res: 0.545385; n = 100; a = 8.389056; b = 3.718282

step: 18; curr res: 0.575279; n = 100; a = 8.389056; b = 3.718282

step: 19; curr res: 0.604954; n = 100; a = 8.389056; b = 3.718282

step: 20; curr res: 0.634415; n = 100; a = 8.389056; b = 3.718282

Метод прямоугольников вернул результат: 2.49999102784388 < настоящего результата

\section{Численные методы: метод касательных}
\begin{sagesilent}
y0 = sqrt(2^cos(x/5) - cos(x/2)) - 1.2
y1 = y0 - (y0)/diff(y0, x)
y2 = y1 - (y1)/diff(y1, x)
y3 = y2 - (y2)/diff(y2, x)
y4 = y3 - (y3)/diff(y3, x)

plot(y0, -10, 10) + list_plot([(y1, 0), (y2, 0), (y3, 0), (y4, 0)], size=30, rgbcolor="red")

\end{sagesilent}

Дана функция: $\sage{y0}$

\sageplot{plot(y0, -10, 10) + list_plot([(y1, 0), (y2, 0), (y3, 0), (y4, 0)], size=30, rgbcolor="red")}
По теореме Канторовича условиями сходимости метода является существование констант A, B, C. Вычислим их:

$|f′(x)| < A$ на $[a; b]$, следовательно, существует производная функции, не равная нулю. $|f′(x)f(x)| < B$ на $[a; b]$, следовательно,
функция ограничена. Существует $f′′(x)$ на $[a; b]$ и
$|f′′(x)| <= C <= 12AB$.

Возьмем отрезок $[−6; −3.8]$:

$A =0.66 > 0.6587160242304654$

$B =1143760>1143759.7680815235$

$C =4.58 >= 4.579451597720261$

$12AB = false$ поэтому, этот метод не может гарантировать сходимость на этом промежутке.

\newpage
\section{Список литературы}
\begin{enumerate}
\item{Справочник по математике, Корн Г., Корн Т., 1973}
\item{sage.org}
\item{wikipedia.org}
\end{enumerate}


\end{document}